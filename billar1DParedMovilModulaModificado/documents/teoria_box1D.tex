\documentclass[a4paper,10pt]{article}
\usepackage[utf8]{inputenc}
\usepackage{amsmath,amssymb}
\usepackage{graphicx}
\usepackage{xcolor}
\usepackage{geometry}
\geometry{margin=2cm}

\title{Documentación: Simulación de Partícula en Caja 1D con Pared Móvil e Inelasticidad}
\author{Proyecto Físico-Computacional}
\date{Octubre 2025}

\begin{document}
\maketitle

\section{Planteamiento del problema}
Se simula una partícula de masa $m$ que se mueve dentro de una caja unidimensional con una pared derecha móvil:
\[
x_R(t) = L_0 - u t
\]
donde $L_0$ es la longitud inicial de la caja y $u$ la velocidad de la pared derecha.

La partícula puede chocar con:
\begin{itemize}
    \item \textbf{Pared izquierda (fija):} colisión inelástica con coeficiente de restitución $e$.
    \item \textbf{Pared derecha (móvil):} colisión inelástica modificada según la velocidad de la pared.
\end{itemize}

\section{Condiciones de frontera}
La velocidad de la partícula después del choque se actualiza según:
\[
\begin{cases}
x < 0 &\Rightarrow v \to - e \, v, \\
x > x_R(t) &\Rightarrow v \to - e \, (v - 2 u),
\end{cases}
\]
donde $0 < e \le 1$ es el \textbf{coeficiente de restitución}.

\section{Parámetros de entrada}
El usuario debe proporcionar:
\begin{itemize}
    \item $x_0$ : posición inicial de la partícula
    \item $v_0$ : velocidad inicial
    \item $u$ : velocidad de la pared derecha
    \item $e$ : coeficiente de restitución
    \item $t_0$ y $t_f$ : tiempo inicial y final de la simulación
    \item $\Delta t$ : paso temporal de integración
\end{itemize}

\section{Método de simulación}
La simulación se realiza mediante integración explícita paso a paso:
\[
x_{n+1} = x_n + v_n \Delta t, \quad t_{n+1} = t_n + \Delta t
\]

En cada paso se verifica:
\begin{enumerate}
    \item Colisión con la pared izquierda
    \item Colisión con la pared derecha móvil
    \item Actualización de la posición de la pared derecha
    \item Cálculo de la energía cinética:
    \[
    E_k = \frac{1}{2} m v^2
    \]
\end{enumerate}

Se registran los valores en un archivo de salida para posteriores gráficas y animaciones.

\section{Gráficas y animaciones generadas}
\begin{itemize}
    \item Posición de la partícula y de la pared móvil vs. tiempo
    \item Energía cinética de la partícula vs. tiempo
    \item Animación GIF combinando ambas gráficas mostrando la evolución temporal
\end{itemize}

\section{Resultados y observaciones}
\begin{itemize}
    \item La energía cinética se conserva si $e=1$ y disminuye en colisiones inelásticas.
    \item Con $0<e<1$, la partícula eventualmente se detiene tras múltiples colisiones.
    \item La animación ilustra claramente el efecto de la inelasticidad en los rebotes.
    \item Se puede incluir un contador de colisiones para cuantificar la disipación.
\end{itemize}

\section{Conclusión}
Este modelo permite estudiar:
\begin{itemize}
    \item Efectos de colisiones inelásticas en un sistema simple.
    \item Disminución progresiva de la energía cinética según el coeficiente de restitución.
    \item Comportamiento dinámico de partículas confinadas con paredes móviles.
\end{itemize}

\end{document}
